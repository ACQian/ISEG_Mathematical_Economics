\chapter{Differential Equations}

Differential equation is an equation involving derivatives. 
$3u+2=5$ is an algebraic equation and its solution is a number: $u=1$.
An example of a differential equation is
\begin{equation*}
    y' = y\Leftrightarrow y(u)=e^u
\end{equation*}
The solution is a \textbf{map}. What is the map such that its derivative is equal to itself?\\

The goals are the following:
\begin{enumerate}
    \item Solve some differential equations
    \item In the cases I cannot solve the differential equations, study the properties of the solutions
\end{enumerate}

Differential equations can be classified as
\begin{itemize}
    \item ordinary:\\
    \begin{equation*}
        y''+y=u^4
    \end{equation*}
    \item partial derivatives $(u_1, u_2)$
    \begin{equation*}
        \frac{\partial y}{\partial u_1} + \frac{\partial^2 y}{\partial u_2^2} = y
    \end{equation*}
\end{itemize}

Moreover, differential equations can be classified according to their order, i.e. the highest degree of derivative appearing in the differential equation.

\ex[]{Classifications}{
    \begin{enumerate}
        \item $y'=y\quad$ ODE of order 1
        \item $y''+u^5y = y\quad$ ODE of order 2
        \item $\frac{\partial f}{\partial u\partial y}=f(u,y)\quad$ PDE of order 2
    \end{enumerate}
}

The \textbf{solution} of a differential equation is a map such that if we substitute that map in the differential equation,
we get a true proportion.
\ex[]{Solution of a DE}{
    $y(u)$
    \begin{equation*}
        y' = 3y\quad y'y'(u)\quad \dot{y}=y'(u)
    \end{equation*}
}

In general, given a differential equation, we get infinitely many solutions. 
But the \textbf{Initial Value Problem} (IVP) has a \textbf{unique solution}, this solution is defined in an open neighborhood of $t_0=0$.
    \begin{equation*}\text{IVP} =
        \begin{cases*}
            g' = e^{3u}\\
            y(0) = 1
        \end{cases*}
    \end{equation*}
    General solution is given by,
    \begin{align*}
        y(u) & = ce^{3u}\, ,c\in\bbR\\
        1 & = ce^{3\times 0}\\
        c & = 1
    \end{align*}
    Thus we just have one solution,
    \begin{equation*}
        y(u) = e^{3u}\, ,u\in\bbR
    \end{equation*}
\ex[]{IVP}{
    \begin{enumerate}
        \item $y'=3$
        \begin{equation*}
            y(u)=3u+c\, ,\underbrace{c\in\bbR}_{\text{set where the constant lives}}
        \end{equation*}
        \item $
            \begin{cases*}
                y'=3\\
                y'(1)=4
            \end{cases*}$
        \begin{equation*}
            y(u) = 3u+c \Rightarrow 3\times 1 + c = 4 \Leftrightarrow c = 1 \Rightarrow y(u) = 3u+1\, ,u\in\bbR
        \end{equation*}
        \item $y'=u^5$
        \begin{equation*}
            y(u)=\frac{u^6}{6}+c\, ,c\in\bbR
        \end{equation*}
    \end{enumerate}
}
\nt{
    \begin{itemize}
        \item $y' = f(u)$ does not depend on $y$
        \item $y(u) = \int f(u)\, du$
    \end{itemize}
}

\section{Linear differential equations of 1st order: $y(u)$}
General setup of the problem:
\begin{itemize}
    \item $y'+a(u) = b(u)$
    \item $a(u)$ and $b(u)$ are smooth maps
\end{itemize}
For example, $y'+\underbrace{3u+4}_{a(u)} = \underbrace{e^u}_{b(u)}$.\\

The first case is the \textbf{homogeneous} case where $b(u) = 0$.
\begin{align*}
    y'+a(u)y & = 0\\
    y' & = -a(u)y \quad ,y\neq 0\\
    \frac{y'}{y} & = -a(u) \Leftrightarrow\\
    \frac{\partial}{\partial u} \log|y| & = -a(u) \Leftrightarrow\\
    \ln|y| & = -\int a(u)\, du + c\\
    |y| & = e^{-\int a(u)\, du}\cdot K
\end{align*}
\ex[]{Homogeneous}{
    $\begin{cases*}
        y' = 3y\\
        a(u) = -3
    \end{cases*}$
    \begin{equation*}
        \frac{y'}{y} = 3 \Leftrightarrow\int\frac{y'}{y}\, du \Leftrightarrow \log|y| = 3u \Leftrightarrow
        y=e^{3u}\cdot K\; ,K\in\bbR\; ,u\in\bbR
    \end{equation*}
}

The second case is the \textbf{non-homogeneous} case where $b(u)\neq0$.
Let $\mu$ be a positive smooth map: $\underbrace{\mu(t)>0, \forall t\in\bbR}_{\text{integrating factor}}$, then the problem is defined as
\begin{align*}
    \mu y' + \mu a(u) y & = \mu b(u)\\
    (y\mu)' & = y'\mu + y\mu' = \mu b(u)
\end{align*}
The left hand side of the equation could be seen as $(y\mu)'$ if
\begin{equation*}
    \mu a(u) = \mu' \Leftrightarrow \mu' = a(u)\mu \Leftrightarrow \mu = e^{\int e(u)\, du}\cdot K
\end{equation*}
The $t_0$ of the solution then is
\begin{align*}
    \frac{\partial}{\partial u}[y\mu] & = \mu\cdot b(u)\\
    \frac{\partial}{\partial u}[y\cdot e^{\int a(u)\, du}\cdot K] & = e^{\int a(u)\,du}\cdot K \cdot b(u)\\
    y\cdot e^{\int a(u)\, du} & = \int e^{\int a(u)\, du}\cdot b(u)\, du + c\\
    y & = \frac{\int e^{\int a(u)\, du}\cdot b(u)\, du + c}{e^{\int a(u)\, du}}\; ,c\in\bbR
\end{align*}
\ex[]{Non-homogeneous}{
    \begin{enumerate}
        \item $y'-y=2ue^{u^2+u}$
            \begin{align*}
                y(u) & = \frac{\int e^{-1\, du}\cdot 2ue^{u^2+u}+c}{e^{-1\, du}}\\
                & = \frac{\int e^{-u}\cdot 2ue^{u^2+u}+c}{e^{-u}}\\
                & = \frac{\int 2ue^{u^2}+c}{e^{-u}}\\
                & = \frac{e^{u^2}}{e^{-u}}\; ,u\in\bbR\; ,c\in\bbR
            \end{align*}
        \item $4u^2y' + 8uy = -12\sin(3u)$\\
        
            Rewriting the problem into the desired form,
            \begin{equation*}
                y' + \frac{8u}{4u^2}y = -\frac{12\sin(3u)}{4u^2}
            \end{equation*}
            Now focusing on solving for $y$,
            \begin{align*}
                y & = \frac{\int e^{\int\frac{2}{u}\, du}\cdot (-\frac{12\sin(3u)}{4u^2})+c}{e^{\int \frac{2}{u}\, du}}\\
                & = \frac{\int e^{2\log(u)}\cdot (-\frac{12\sin(3u)}{4u^2})+c}{e^{\log (u^2)}}\\
                & = \frac{\int u^2(-\frac{12\sin(3u)}{4u^2})+c}{e^{\log (u^2)}}\\
                & = \frac{\int -3\sin(3u)+c}{u^2}\\
                & = \frac{\cos(3u)+c}{u^2}\; ,c\in\bbR\; ,u\in\bbR\backslash\{0\}
            \end{align*}
    \end{enumerate}
}

In this course, given an IVP, we have just one solution.
\begin{enumerate}
    \item \begin{gather*}
        y' = f(u)\\
        y = \int f(u)\, du + c\, ,c\in\bbR
    \end{gather*}
    \item Linear differential equation of first order
    \begin{equation*}
        y' + a(u)y = b(u)
    \end{equation*}
    \item Separable differential equation
    \begin{gather*}
        y' = \frac{f(u)}{g(y)}\quad ,g(y)\neq 0\quad f,g \text{ differentiable maps} \Leftrightarrow\\
        g(y)y' = f(u)\quad \text{using chain rule }\Leftrightarrow\\
        \frac{\partial}{\partial u}[G(y)] = f(u) \Leftrightarrow\\
        G(y) = F(u) + c\, ,c\in\bbR
    \end{gather*}
    where $G(y) = \int g(y)\, dy$, $F(u) = \int f(u)\, du$, $G' = g$, 
    and $\frac{\partial}{\partial u}(G\circ g)(u) = \frac{\partial G}{\partial y}\cdot \frac{\partial y}{\partial u}(u) = g(y)y'$
\end{enumerate}

\ex[]{Seperable differential equation}{
    \begin{enumerate}
        \item $y' = 2ty$ is separable
        \item $y' = \cos (2ty)$ is not separable
        \item $\begin{cases*}
            u' = -2tu^2\\
            u(1) = -1
        \end{cases*}$
        \begin{gather*}
            u' = -2tu^2\\
            u^{-2}u' = -2t\\
            \frac{u^{-1}}{-1} = -t^2 + c\\
            -\frac{1}{u} = -t^2 + c\\
            u(t) = \frac{1}{t^2-c}
        \end{gather*}
        Now using $t(1) = -1$
        \begin{gather*}
            \frac{1}{1-c} = -1 \Leftrightarrow c = 2\\
            u(t) = \frac{1}{t^2 - 2}
        \end{gather*}
        We now choose the interval that contains $t=1$
        \begin{equation*}
            D = \bbR \backslash \{\pm \sqrt{2}\}
        \end{equation*}
        \item $e^y (4+u^2)y' = u(2+e^y)$
        \begin{gather*}
            \underbrace{\frac{e^y}{2+e^y}}_{g(y)}y' = \underbrace{\frac{u}{4+u^2}}_{f(u)}\\
            \ln(2+e^y) = \frac{1}{2}\ln(4+u^2) + c\\
            2 + e^y = \exp \{\frac{1}{2}\ln(4+u^2)+ c\}\\
            e^y = \exp\{\frac{1}{2}\ln(4+u^2)\}\,e^c - 2\\
            e^y = \sqrt{4+u^2}\, e^c-2\\
            y = \ln(\sqrt{4+u^2}\underbrace{e^c}_{k}-2)
        \end{gather*}
        where $\begin{cases*}
            c\in\bbR\\
            k\in\bbR^+
        \end{cases*}$
    \end{enumerate}
}

\subsection{Exact differential equations}
General set up of the question,
\begin{gather*}
    y(t)\\
    \phi (t,y) = k\, ,k\in\bbR\\
    \phi : \bbR^2\to\bbR\quad c^2 \text{ map}
\end{gather*}
Computing the derivative of $\phi$ with respect to $t$,
\begin{equation*}
    \frac{\partial \phi}{\partial t}\cdot\underbrace{\frac{\partial t}{\partial t}}_{=1} + \frac{\partial \phi}{\partial y}\cdot\frac{\partial y}{\partial t} = 0
    \Leftrightarrow \underbrace{\frac{\partial \phi}{\partial t} + \frac{\partial \phi}{\partial y}\cdot y' = 0}_{\text{form of an exact differential equation}}
\end{equation*}

Given an exact differential equation, the first goal is to write $y(t)$ explicitly. 
If that is not possible, we want to compute $\phi(t,y)$.
In the ladder case, we say taht the solution is implicit.
\begin{equation*}
    \underbrace{\frac{\partial \phi}{\partial t}}_{f(t,y)} + \underbrace{\frac{\partial \phi}{\partial y}}_{g(t,y)} y' = 0
\end{equation*}
\lem[]{}{
    If $f$ and $g$ are map defined in an rectangle of the form $(a,b)\times(c,d)$ where $a,b,c,d \in\bbR$, then the following propositions are equivalent.
    \begin{enumerate}
        \item There exists $\phi$ such that $\begin{cases*}
            \frac{\partial\phi}{\partial t} = f(t,y)\\
            \frac{\partial\phi}{\partial y} = g(t,y)
        \end{cases*}$
        \item $\displaystyle\frac{\partial f}{\partial y} = \frac{\partial g}{\partial t}$
    \end{enumerate}
}

%\dfn{Definition Topic}{Definition Statement}
%\thm{Theorem Name}{Theorem Statement}
%\cor[cori]{Corollary Name}{Corollary Statement}
%\lem{Lemma Name}{Lemma Statement}
%\clm{Claim Name}{Claim Statement}
%\ex{Example Name}{Example explained}
%\opn{Open Question Name}{Question Statement}
%\pr{Question Name}{Question Statement}
%\nt{Special Note}
%\wc{Wrong Concept topic}{Explanation}