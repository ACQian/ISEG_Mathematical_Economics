\chapter{Differential Equations}

Differential equation is an equation involving derivatives. 
$3u+2=5$ is an algebraic equation and its solution is a number: $u=1$.
An example of a differential equation is
\begin{equation*}
    y' = y\Leftrightarrow y(u)=e^u
\end{equation*}
The solution is a \textbf{map}. What is the map such that its derivative is equal to itself?\\

The goals are the following:
\begin{enumerate}
    \item Solve some differential equations
    \item In the cases I cannot solve the differential equations, study the properties of the solutions
\end{enumerate}

Differential equations can be classified as
\begin{itemize}
    \item ordinary:\\
    \begin{equation*}
        y''+y=u^4
    \end{equation*}
    \item partial derivatives $(u_1, u_2)$
    \begin{equation*}
        \frac{\partial y}{\partial u_1} + \frac{\partial^2 y}{\partial u_2^2} = y
    \end{equation*}
\end{itemize}

Moreover, differential equations can be classified according to their order, i.e. the highest degree of derivative appearing in the differential equation.

\ex[]{Classifications}{
    \begin{enumerate}
        \item $y'=y\quad$ ODE of order 1
        \item $y''+u^5y = y\quad$ ODE of order 2
        \item $\frac{\partial f}{\partial u\partial y}=f(u,y)\quad$ PDE of order 2
    \end{enumerate}
}

The \textbf{solution} of a differential equation is a map such that if we substitute that map in the differential equation,
we get a true proportion.
\ex[]{Solution of a DE}{
    $y(u)$
    \begin{equation*}
        y' = 3y\quad y'y'(u)\quad \dot{y}=y'(u)
    \end{equation*}
}

In general, given a differential equation, we get infinitely many solutions. 
But the \textbf{Initial Value Problem} (IVP) has a \textbf{unique solution}, this solution is defined in an open neighborhood of $t_0=0$.
    \begin{equation*}\text{IVP} =
        \begin{cases*}
            g' = e^{3u}\\
            y(0) = 1
        \end{cases*}
    \end{equation*}
    General solution is given by,
    \begin{align*}
        y(u) & = ce^{3u}\, ,c\in\bbR\\
        1 & = ce^{3\times 0}\\
        c & = 1
    \end{align*}
    Thus we just have one solution,
    \begin{equation*}
        y(u) = e^{3u}\, ,u\in\bbR
    \end{equation*}
\ex[]{IVP}{
    \begin{enumerate}
        \item $y'=3$
        \begin{equation*}
            y(u)=3u+c\, ,\underbrace{c\in\bbR}_{\text{set where the constant lives}}
        \end{equation*}
        \item $
            \begin{cases*}
                y'=3\\
                y'(1)=4
            \end{cases*}$
        \begin{equation*}
            y(u) = 3u+c \Rightarrow 3\times 1 + c = 4 \Leftrightarrow c = 1 \Rightarrow y(u) = 3u+1\, ,u\in\bbR
        \end{equation*}
        \item $y'=u^5$
        \begin{equation*}
            y(u)=\frac{u^6}{6}+c\, ,c\in\bbR
        \end{equation*}
    \end{enumerate}
}
\nt{
    \begin{itemize}
        \item $y' = f(u)$ does not depend on $y$
        \item $y(u) = \int f(u)\, du$
    \end{itemize}
}

\section{Linear differential equations of 1st order: $y(u)$}
General setup of the problem:
\begin{itemize}
    \item $y'+a(u) = b(u)$
    \item $a(u)$ and $b(u)$ are smooth maps
\end{itemize}
For example, $y'+\underbrace{3u+4}_{a(u)} = \underbrace{e^u}_{b(u)}$.\\

The first case is the \textbf{homogeneous} case where $b(u) = 0$.
\begin{align*}
    y'+a(u)y & = 0\\
    y' & = -a(u)y \quad ,y\neq 0\\
    \frac{y'}{y} & = -a(u) \Leftrightarrow\\
    \frac{\partial}{\partial u} \log|y| & = -a(u) \Leftrightarrow\\
    \ln|y| & = -\int a(u)\, du + c\\
    |y| & = e^{-\int a(u)\, du}\cdot K
\end{align*}
\ex[]{Homogeneous}{
    $\begin{cases*}
        y' = 3y\\
        a(u) = -3
    \end{cases*}$
    \begin{equation*}
        \frac{y'}{y} = 3 \Leftrightarrow\int\frac{y'}{y}\, du \Leftrightarrow \log|y| = 3u \Leftrightarrow
        y=e^{3u}\cdot K\; ,K\in\bbR\; ,u\in\bbR
    \end{equation*}
}

The second case is the \textbf{non-homogeneous} case where $b(u)\neq0$.
Let $\mu$ be a positive smooth map: $\underbrace{\mu(t)>0, \forall t\in\bbR}_{\text{integrating factor}}$, then the problem is defined as
\begin{align*}
    \mu y' + \mu a(u) y & = \mu b(u)\\
    (y\mu)' & = y'\mu + y\mu' = \mu b(u)
\end{align*}
The left hand side of the equation could be seen as $(y\mu)'$ if
\begin{equation*}
    \mu a(u) = \mu' \Leftrightarrow \mu' = a(u)\mu \Leftrightarrow \mu = e^{\int e(u)\, du}\cdot K
\end{equation*}
The $t_0$ of the solution then is
\begin{align*}
    \frac{\partial}{\partial u}[y\mu] & = \mu\cdot b(u)\\
    \frac{\partial}{\partial u}[y\cdot e^{\int a(u)\, du}\cdot K] & = e^{\int a(u)\,du}\cdot K \cdot b(u)\\
    y\cdot e^{\int a(u)\, du} & = \int e^{\int a(u)\, du}\cdot b(u)\, du + c\\
    y & = \frac{\int e^{\int a(u)\, du}\cdot b(u)\, du + c}{e^{\int a(u)\, du}}\; ,c\in\bbR
\end{align*}
\ex[]{Non-homogeneous}{
    \begin{enumerate}
        \item $y'-y=2ue^{u^2+u}$
            \begin{align*}
                y(u) & = \frac{\int e^{-1\, du}\cdot 2ue^{u^2+u}+c}{e^{-1\, du}}\\
                & = \frac{\int e^{-u}\cdot 2ue^{u^2+u}+c}{e^{-u}}\\
                & = \frac{\int 2ue^{u^2}+c}{e^{-u}}\\
                & = \frac{e^{u^2}}{e^{-u}}\; ,u\in\bbR\; ,c\in\bbR
            \end{align*}
        \item $4u^2y' + 8uy = -12\sin(3u)$\\
        
            Rewriting the problem into the desired form,
            \begin{equation*}
                y' + \frac{8u}{4u^2}y = -\frac{12\sin(3u)}{4u^2}
            \end{equation*}
            Now focusing on solving for $y$,
            \begin{align*}
                y & = \frac{\int e^{\int\frac{2}{u}\, du}\cdot (-\frac{12\sin(3u)}{4u^2})+c}{e^{\int \frac{2}{u}\, du}}\\
                & = \frac{\int e^{2\log(u)}\cdot (-\frac{12\sin(3u)}{4u^2})+c}{e^{\log (u^2)}}\\
                & = \frac{\int u^2(-\frac{12\sin(3u)}{4u^2})+c}{e^{\log (u^2)}}\\
                & = \frac{\int -3\sin(3u)+c}{u^2}\\
                & = \frac{\cos(3u)+c}{u^2}\; ,c\in\bbR\; ,u\in\bbR\backslash\{0\}
            \end{align*}
    \end{enumerate}
}

In this course, given an IVP, we have just one solution.
\begin{enumerate}
    \item \begin{gather*}
        y' = f(u)\\
        y = \int f(u)\, du + c\, ,c\in\bbR
    \end{gather*}
    \item Linear differential equation of first order
    \begin{equation*}
        y' + a(u)y = b(u)
    \end{equation*}
    \item Separable differential equation
    \begin{gather*}
        y' = \frac{f(u)}{g(y)}\quad ,g(y)\neq 0\quad f,g \text{ differentiable maps} \Leftrightarrow\\
        g(y)y' = f(u)\quad \text{using chain rule }\Leftrightarrow\\
        \frac{\partial}{\partial u}[G(y)] = f(u) \Leftrightarrow\\
        G(y) = F(u) + c\, ,c\in\bbR
    \end{gather*}
    where $G(y) = \int g(y)\, dy$, $F(u) = \int f(u)\, du$, $G' = g$, 
    and $\frac{\partial}{\partial u}(G\circ g)(u) = \frac{\partial G}{\partial y}\cdot \frac{\partial y}{\partial u}(u) = g(y)y'$
\end{enumerate}

\ex[]{Seperable differential equation}{
    \begin{enumerate}
        \item $y' = 2ty$ is separable
        \item $y' = \cos (2ty)$ is not separable
        \item $\begin{cases*}
            u' = -2tu^2\\
            u(1) = -1
        \end{cases*}$
        \begin{gather*}
            u' = -2tu^2\\
            u^{-2}u' = -2t\\
            \frac{u^{-1}}{-1} = -t^2 + c\\
            -\frac{1}{u} = -t^2 + c\\
            u(t) = \frac{1}{t^2-c}
        \end{gather*}
        Now using $t(1) = -1$
        \begin{gather*}
            \frac{1}{1-c} = -1 \Leftrightarrow c = 2\\
            u(t) = \frac{1}{t^2 - 2}
        \end{gather*}
        We now choose the interval that contains $t=1$
        \begin{equation*}
            D = \bbR \backslash \{\pm \sqrt{2}\}
        \end{equation*}
        \item $e^y (4+u^2)y' = u(2+e^y)$
        \begin{gather*}
            \underbrace{\frac{e^y}{2+e^y}}_{g(y)}y' = \underbrace{\frac{u}{4+u^2}}_{f(u)}\\
            \ln(2+e^y) = \frac{1}{2}\ln(4+u^2) + c\\
            2 + e^y = \exp \{\frac{1}{2}\ln(4+u^2)+ c\}\\
            e^y = \exp\{\frac{1}{2}\ln(4+u^2)\}\,e^c - 2\\
            e^y = \sqrt{4+u^2}\, e^c-2\\
            y = \ln(\sqrt{4+u^2}\underbrace{e^c}_{k}-2)
        \end{gather*}
        where $\begin{cases*}
            c\in\bbR\\
            k\in\bbR^+
        \end{cases*}$
    \end{enumerate}
}

\section{Exact differential equations}
General set up of the question,
\begin{gather*}
    y(t)\\
    \phi (t,y) = k\, ,k\in\bbR\\
    \phi : \bbR^2\to\bbR\quad c^2 \text{ map}
\end{gather*}
Computing the derivative of $\phi$ with respect to $t$,
\begin{equation*}
    \frac{\partial \phi}{\partial t}\cdot\underbrace{\frac{\partial t}{\partial t}}_{=1} + \frac{\partial \phi}{\partial y}\cdot\frac{\partial y}{\partial t} = 0
    \Leftrightarrow \underbrace{\frac{\partial \phi}{\partial t} + \frac{\partial \phi}{\partial y}\cdot y' = 0}_{\text{form of an exact differential equation}}
\end{equation*}

Given an exact differential equation, the first goal is to write $y(t)$ explicitly. 
If that is not possible, we want to compute $\phi(t,y)$.
In the ladder case, we say taht the solution is implicit.
\begin{equation*}
    \underbrace{\frac{\partial \phi}{\partial t}}_{f(t,y)} + \underbrace{\frac{\partial \phi}{\partial y}}_{g(t,y)} y' = 0
\end{equation*}
\lem[]{}{
    If $f$ and $g$ are map defined in an rectangle of the form $(a,b)\times(c,d)$ where $a,b,c,d \in\bbR$, then the following propositions are equivalent.
    \begin{enumerate}
        \item There exists $\phi$ such that $\begin{cases*}
            \frac{\partial\phi}{\partial t} = f(t,y)\\
            \frac{\partial\phi}{\partial y} = g(t,y)
        \end{cases*}$
        \item $\displaystyle\frac{\partial f}{\partial y} = \frac{\partial g}{\partial t}$
    \end{enumerate}
}

\ex[]{}{
    $\begin{cases*}
        \frac{2u}{y^3}+\frac{y^2-3u^2}{y^4}\,\frac{dy}{du} = 0\\
        y(0) = 1
    \end{cases*}$\\

    In this case, we can use the Lemma proposed earlier where $\displaystyle\frac{\partial f}{\partial y} = \frac{\partial g}{\partial u}$ 
    if $\exists \;\phi$: satisfying the hypothesis.
    \begin{gather*}
        \underbrace{\frac{2u}{y^3}}_{\frac{\partial \phi}{\partial t}=f(u,y)} + \underbrace{\frac{y^2-3u^2}{y^4}}_{\frac{\partial\phi}{\partial y}= g(u,y)}y' = 0\\
        f(u,y) = \frac{2u}{y^3} = 2uy^{-3} \Rightarrow \frac{\partial f}{\partial y}(u,y) = -6uy^{-4}\\
        g(u,y) = \frac{y^2-3u^2}{y^4} = \frac{1}{y^2}-3u^2y^{-4} \Rightarrow \frac{\partial g}{\partial u}(u,y) = -6uy^{-4}
    \end{gather*}
    Hypothesis for the Lemma is satisfied, thus,
    \begin{equation*}
        \frac{\partial f}{\partial u}(u,y) = 2uy^{-3} 
        \Rightarrow \phi(u,y) = u^2y^{-3}+h(y) 
        \Rightarrow \frac{\partial\phi}{\partial y}(u,y) = -3u^2y^{-4}+h^2(y)
        \Rightarrow h'(y) = \frac{1}{y^2}
        \Rightarrow h(y) = -\frac{1}{y}+k
    \end{equation*}
    Solution:
    \begin{gather*}
        \underbrace{\phi(u,y)}_{y(0)=1} = u^2y^{-3}-\frac{1}{y}+k
        \Leftrightarrow \underbrace{\phi(0,1)}_{=0} = -\frac{1}{1}=k
        \Leftrightarrow k = 1\\
        \therefore \phi(u,y) = u^2y^{-3}-\frac{1}{y} + 1
    \end{gather*}
}

\section{Linear differential equations with constant coefficient of 2nd degree}
General set up of the question:
\begin{equation*}
    ay''(u) + by'(u) + cy(u) = f(u)\quad a,b,c\in\bbR,\quad f(u)\text{ is a differentiable map}
\end{equation*}

\subsection{Homogeneous case: $f(u) = 0$}
\begin{equation}
    ay'' + by' + cy = 0
\end{equation}
\dfn[]{}{
    The characteristic polynomial associated to Eq.(3.1) is $a\lambda^2 + b\lambda + c$.\\

    The characteristic equation associated to Eq.(3.1) is $\underbrace{a\lambda^2+b\lambda+c=0}_{P(\lambda)}$. The graph of $P(\lambda)$ is a parabola.
}

Couple \textbf{propositions}:
    \begin{itemize}
        \item If $P(\lambda) = 0$ has two different solutions, say $\lambda_1$ and $\lambda_2$, then the general solution of Eq.(3.1) is
        \begin{equation*}
            y(u) = c_1e^{\lambda_1u}+c_2e^{\lambda_2u}\quad c_1,c_2\in\bbR
        \end{equation*}
        \item If $P(\lambda) = 0$ has a unique real solution, $\lambda = \lambda_0$, then the solution of Eq.(3.1) is
        \begin{equation*}
            y(u) = (c_1+c_2u)e^{\lambda u}\quad c_1,c_2\in\bbR
        \end{equation*}
        \item If $P(\lambda) = 0$ has complex and non-real solutions, say $\alpha \pm i\omega$, then the solution of Eq.(3.1) is
        \begin{equation*}
            y(u) = e^{\lambda n}[c_1\cos(\omega u)+c_2\sin(\omega n)]\quad c_1,c_2\in\bbR
        \end{equation*}
    \end{itemize}

\ex[]{}{
    \begin{enumerate}
        \item $y'' - 3y' + 2y = 0$
        \begin{gather*}
            P(\lambda) = \lambda^2 -3\lambda +2\\
            P(\lambda) = 0 \Leftrightarrow \lambda = 1 \vee \lambda = 2
        \end{gather*}
        General solution:
        \begin{equation*}
            y(u) = c_1 e^u + c_2 e^{2u}
        \end{equation*}
        Result:
        \begin{gather*}
            y'' -3y' = 0\\
            P(\lambda) = \lambda^2 -3\lambda\\
            \alpha = 0\quad \omega = 3
        \end{gather*}
        \item $y'' + 9y = 0$
        \begin{gather*}
            P(\lambda) = \lambda^2 + 9\\
            P(\lambda) = 0\\
            \lambda = \pm 3i\\
            y(u) = c_1\cos(3u)+c_2\sin(3u)\quad c_1,c_2\in\bbR
        \end{gather*}
    \end{enumerate}
}
\textbf{Remark}:
\begin{itemize}
    \item 1st degree $\begin{cases*}
    y'=y\\
    y(t_0) = y_0
    \end{cases*}$
    \item 2nd degree $\begin{cases*}
    ay'' + by' + cy = 0\\
    y(t_0) = y_0\, ,y(t_1)=y_1
    \end{cases*} \rightarrow $ unique solution
\end{itemize}

\subsection{Non-homogenous case: $f(u)\neq 0$}
\begin{equation}
    ay'' + by'' + cy = f(u)
\end{equation}
To define some terminology that would be used:
\begin{itemize}
    \item $y_{part}(u) = $ particular solution of Eq.(3.2)
    \item $y_{hom}(u) = $ solution of the homogeneous equation associated to Eq.(3.2)
\end{itemize}
\lem[]{}{
    The general solution of Eq.(3.2) is of the form
    \begin{equation*}
        y(u) = y_{hom}(u) + y_{part}(u)
    \end{equation*}
}
\begin{proof}
    \begin{gather*}
        y'(u) = y_{hom}'(u) + y_{part}'(u)\\
        y''(u) = y_{hom}''(u) + y_{part}''(u)
    \end{gather*}
    \begin{align*}
        ay''(u) + by'(u) + cy(u) 
        & = a[y''_{hom}(u)+y_{part}''(u)] + b[y_{hom}'(u)+y_{part}'(u)] + c[y_{hom}(u)+y_{part}(u)]\\
        & = \underbrace{(ay''_{hom}(u)+by'_{hom}(u)+cy_{hom}(u))}_{=0} + \underbrace{(ay_{part}''(u)+by_{part}''(u)+cy_{part}(u))}_{f(u)}\\
        & = f(u)
    \end{align*}
\end{proof}
The difficulty now is to obtain a particular solution of Eq.(3.2). For this, we are going to use the following table,
\begin{table}[h!]
    \centering
    \begin{tabular}{ll}
        \hline
        $f(u)$ & $y_{\text{part}}(u)$ \\ \hline
        $c$ & $k$ \\
        $ce^{au}$ & $ke^{au}$ \\
        $c\cos(bu)$ & $k_1\cos(bu)+k_2\sin(bu)$ \\
        $c\cos(bu)e^{au}$ & $k_1e^{au}\cos(bu) + k_2e^{au}\sin(bu)$ \\
        $c_2u^2+c_1u+c_0$ & $k_2u^2+k_1u+k_0$ \\ \hline
    \end{tabular}
    \caption{Typical forms of $f(u)$ and corresponding $y_{\text{part}}(u)$ in non-homogeneous ODEs}
    \label{tab:ypart}
\end{table}
\ex[]{}{
    $y''-4y'+4y = 0$\\

    Homogeneous:
    \begin{gather*}
        y'' -4y'+4y = 0\\
        \lambda^2 - 4\lambda + 4 = 0\\
        \lambda = \frac{4\pm\sqrt{16-4\times 1\times 4}}{2}\\
        \lambda = 2\rightarrow \text{one solution}\\
        y(u) = (c_1+c_2 u)e^{2u}\quad c_1,c_2\in\bbR
    \end{gather*}
    Particular solution:
    \begin{gather*}
        y_{part}(u) = Au^2 + Bu + C\\
        y_{part}'(u) = 2Au + B\\
        y''(u) = 2A\\
        \underbrace{2A}_{y''} - 4\underbrace{(2Au+B)}_{y'} + 4\underbrace{(Au^2+Bu+C)}_{y} = 8u^2\\
        (2A - 4B + 4C) + (-8A + 4B)u + 4Au^2 = 8u^2\\
        \begin{cases}
            4A = 8\\
            -8A+4B = 0\\
            2A-4B+4C = 0
        \end{cases} \Leftrightarrow \begin{cases}
            A = 2\\
            B = 4\\
            C = 3
        \end{cases}
    \end{gather*}
    The general solution is 
    \begin{equation*}
        y(u) = (c_1+c_2u)e^{2u} + 2u^2+4u+3\quad c_1,c_2\in\bbR
    \end{equation*}
}

\section{Models of population}
First approach: $p(t)$ which models the population of instant $t$, $t\in\bbR_0^+$, and $p(T)$ a differentiable map.\\

Consider the situation where $\begin{cases*}
    r_B: \text{ birth rate}\\
    r_D: \text{ death rate}
\end{cases*}$. The differentiable euqation that models the evolution of the popualtion is given by
\begin{equation}
    p'(t) = \underbrace{[r_B - r_D]}_{K}\cdot p(t)
\end{equation}
where Eq.(3.3) could be written as $p'=Kp$ which is also known as the \textbf{Malthus Law}. The general solution is 
\begin{equation*}
    p(t) = p_0 \cdot e^{Kt}\quad ,p\in\bbR_0^+
\end{equation*}\\

Assuming $K>0$, the the graph of $p$ is 
\begin{center}
    \includegraphics[scale=0.3]{Images/30.png}\\
    $p' = ap -bp^2\text{ Logistic Law}\quad a>b>0$
\end{center}
In this case, $\lim_{t\to+\infty}p(t) = +\infty$. This model does not contemplate completion.\\

Separable solution:
\begin{equation*}
    p' = p(a-bp) \Leftrightarrow \frac{p'}{p(a-bp)} = 1
\end{equation*}
We should find
\begin{gather*}
    \int \frac{1}{p(a-bp)}\, dp\quad\text{where } \frac{1}{p(a-bp)} = \frac{A}{p}+\frac{B}{(a-bp)}
\end{gather*}
\begin{gather*}
    A(a-bp)+Bp =1 \Leftrightarrow\\
    Aa -Abp + Bp = 1 \Leftrightarrow\\
    \begin{cases*}
        p(-Ab+B) = 0\\
        Aa = 1
    \end{cases*} \Leftrightarrow\\
    \begin{cases*}
        -Ab + B = 0\\
        A = \frac{1}{a}
    \end{cases*} \Leftrightarrow\\
    -\frac{b}{a}+B = 0 \Leftrightarrow\\
    \begin{cases*}
        B = \frac{b}{a}\\
        A = \frac{1}{a}
    \end{cases*}\Leftrightarrow
\end{gather*}
Thus we have,
\begin{equation*}
    \frac{1}{p(a-bp)} = \frac{\frac{1}{a}}{p} + \frac{\frac{b}{a}}{(a-bp)}
\end{equation*}
Taking the derivative,
\begin{align*}
    \int\frac{1}{p(a-bp)}\, dp
    & = \int \frac{\frac{1}{a}}{p}\, dp + \int \frac{\frac{b}{a}}{a-bp}\, dp \\
    & = \frac{1}{a}\ln(p) + \frac{b}{a}\cdot\frac{1}{-b}\ln[a-bp]\\
    & = \frac{1}{a}\ln(p) -\frac{1}{a}\ln|a-bp|
\end{align*}
Coming back to the differentiable equation, one gets,
\begin{gather*}
    \frac{1}{a}\ln(p) - \frac{1}{a}\ln|a-bp| = t + c \Leftrightarrow\\
    \frac{1}{a}\ln|\frac{p}{a-bp}| = t + c \Leftrightarrow\\
    \ln|\frac{p}{a-bp}| = at + c \Leftrightarrow\\
    |\frac{p}{a-bp}| = e^{at}\cdot c\, c\in\bbR_0^+ \Leftrightarrow\\
    |\frac{a-bp}{p}| = \frac{1}{e^{at}\cdot c} \underset{(a-bp)>0}{\Leftrightarrow}\\
    \frac{a}{p}-b = \frac{1}{e^{at}\cdot c}\Leftrightarrow\\
    \frac{a}{p} = b + \frac{1}{e^{at}\cdot c}\Leftrightarrow\\
    p(t) = \frac{a}{b + \frac{1}{e^{at}\cdot c}}
\end{gather*}
\begin{center}
    \includegraphics[scale=0.4]{Images/31.png}\\
    \begin{equation*}
        \lim_{t\to+\infty}p(t) = \frac{a}{b}\to \text{ carrying capacity}
    \end{equation*}
\end{center}

\section{Linear system of differential equation}
The typology space considered is $\bbR^n$ with $(x_1,\ldots,x_n) \in\bbR^n$.
\begin{equation}
    \begin{cases}
        \dot{x}_1 = a_{11}x_1 + a_{12}x_2+\ldots+a_{1n}x_n\\
        \vdots\\
        \dot{x}_n = a_{n1}x_1 + a_{n2}x_2 + \ldots + a_{nn}x_n
    \end{cases}
\end{equation}
where $\dot{x}_i = \frac{\partial x_i}{\partial t}$.

\ex[]{}{
    $\begin{cases*}
        \dot{x}_1 = x_1\\
        \dot{x}_2 = 3x_2
    \end{cases*}$
    \begin{equation*}
        \dot{x}_1 = \frac{\partial x_1}{\partial t}
    \end{equation*}
    $(x_1(t), x_2(t)) = (e^{t}, e^{3t})$ is a solution.
    \begin{gather*}
        \dot{x}_1(t) = e^{t} = x_1(t)\\
        \dot{x}_2(t) = 3e^{3t} = 3x_2(t)
    \end{gather*}
}
\dfn[]{n-uple}{
    A solution of Eq. (3.4) is a n-uple $(x_1(t),\ldots,x_n(t))\in\bbR^n$ that satisfies the equalities.
}
\ex[]{}{
    $\begin{cases*}
        \dot{x}_1 = 3\underbrace{x_1^2}_{\text{this is not linear}}\\
        \dot{x}_2 = 4x_1 + 8x_2
    \end{cases*}$
}

This goal is to solve planer systems of differential equations.
In other words, the main goal is to solve:
\begin{equation}
    \begin{cases*}
        \dot{x} = ax + by\\
        \dot{y} = xc + dy
    \end{cases*}
\end{equation}
As before, Eq. (3.5) has infinitely many solutions.\\

The \textbf{first step} is to write Eq. (3.5) in a matrix form.
\begin{equation*}
    \left(\begin{array}{c}
        \dot{x}\\
        \dot{y}
    \end{array}\right) = \underbrace{\left(\begin{array}{cc}
        a & b\\
        c & d
    \end{array}\right)}_{A}\left(\begin{array}{c}
        x\\
        y
    \end{array}\right)
\end{equation*}

The \textbf{second step} is to check the eigenvalues and eigenvectors of $A$.
\nt{
    $P(\lambda) = \det (A-\lambda Id) \to$ characteristic polynomial\\
    $P(\lambda) = 0 \to$ characteristic equation\\
    $P(\lambda) = (\lambda - \lambda_1)(\lambda - \lambda_2)^*$ where * is the algebraic multiplicity of $\lambda_2$\\
    $E_{\lambda} = <(x,y,z),(\alpha,\beta,\theta)>$ where dim $E_\lambda$ is the geometric multiplicity of $\lambda$
}

\textbf{First case}: $A$ has two real eigenvalues $\lambda_1, \lambda_2$ associated to the eigenvectors $v_1, v_2$ respectively.
\begin{equation*}
    \left(\begin{array}{c}
        x(t)\\
        y(t)
    \end{array}\right) = k_1 e^{\lambda_1 t}v_1 + k_2 e^{\lambda_2 t}v_2\quad k_1,k_2\in\bbR
\end{equation*}
\ex[]{}{
    $\begin{cases*}
        \dot{x} = 3x-y\\
        \dot{y} = 4x-2y
    \end{cases*}$
    \begin{equation*}
        A = \left[\begin{array}{cc}
            3 & -1\\
            4 & -2
        \end{array}\right]
    \end{equation*}
    \begin{align*}
        P(\lambda)
        & = det\left(\begin{array}{cc}
            3-\lambda & -1\\
            4 & -2-\lambda
        \end{array}\right)\\
        & = (3-\lambda)(-2-\lambda)+4\\
        & = -6 - 3\lambda + 2\lambda + \lambda^2 + 4\\
        & = \lambda^2 -\lambda -2\\
        & = (\lambda +1)(\lambda -2)
    \end{align*}
    Eigenvalues: $-1, 2$\\

    Now finding the eigenvectors:
    \begin{itemize}
        \item $E_{-1}$
        \begin{equation*}
            \left(\begin{array}{cc}
                4 & -1\\
                4 & -1
            \end{array}\right)\left(\begin{array}{c}
                x\\
                y
            \end{array}\right) = \left(\begin{array}{c}
                0\\
                0
            \end{array}\right) \Leftrightarrow 
            4x - y = 0 \Leftrightarrow
            y = 4x \Rightarrow E_{-1} : \,<(1,4)>
        \end{equation*}
        \item $E_{2}$
        \begin{equation*}
            (\cdots) \Rightarrow E_{2} :\, <(1,1)>
        \end{equation*}
    \end{itemize}
    So the general solution is:
    \begin{equation*}
        \left(\begin{array}{c}
            x(t)\\
            y(t)
        \end{array}\right) = k_1e^{-t}\left(\begin{array}{c}
            1\\
            4
        \end{array}\right) + k_2e^{2t}\left(\begin{array}{c}
            1\\
            1
        \end{array}\right)\quad k_1,k_2\in\bbR
    \end{equation*}
}

\textbf{Second case}: $A$ has a single eigenvalue $\lambda\in\bbR$ with algebraic multiplicity equal to geometric multiplicity = 2, i.e. $\lambda, E_{\lambda}=<v_1, v_2>$.
\begin{equation*}
    \left(\begin{array}{c}
        x(t)\\
        y(t)
    \end{array}\right) = k_1 e^{\lambda t}v_1 + k_2 e^{\lambda t}v_2\quad k_1,k_2\in\bbR
\end{equation*}

\textbf{Third case}: $A$ has a single eigenvalue $\lambda$ and its algebraic multiplicity is bigger than its geometric multiplicity, i.e. $a.m.(\lambda) = 2 > g.m.(\lambda) =1$. 
We get one eigenvalue for free: $E_{\lambda} = <v_1>$. In this case, we  need to find an $\omega$ such that 
\begin{equation*}
    \begin{cases*}
        (A-\lambda Id)^2\vec{\omega} = \vec{0}\\
        A\vec{\omega} \neq \lambda\vec{\omega} \to \omega\text{ cannot be eigenvector}
    \end{cases*}
\end{equation*}
We then get the general solution,
\begin{equation*}
    X(t) = k_1e^{\lambda t}v_1 + k_2e^{\lambda t}(\omega + t(A-\lambda Id)\omega)\quad k_1,k_2\in\bbR
\end{equation*}

\textbf{Fourth case}: $A$ has a pair of complex non-real eigenvalues in the form $\alpha\pm i\omega$. First compute the eigenvectors for $v_1\in\bbC$. Then the generation solution would be,
\begin{equation*}
    X(t) = \left(\begin{array}{c}
        x(t)\\
        y(t)
    \end{array}\right) = e^{(\alpha + i\omega)t}v_1
\end{equation*}
\begin{align*}
    e^{(\alpha+i\omega)t}
    & = e^{\alpha t}\cdot e^{i\omega t}\quad\text{using the Euler rotation}\\
    & = e^{\alpha t}(\cos(\omega t)+i\sin(\omega t))
\end{align*}
\begin{equation*}
    v_1\in\bbC \Leftrightarrow v_1 = \underbrace{\omega_1}_{Re(v_1)} + i\underbrace{\omega_2}_{Im(v_1)}
\end{equation*}

\lem[]{}{
    If $\phi(t)\in\bbC$ is a solution of $\dot{x} = Ax$, then $Re(\phi(t))$ and $Im(\phi(t))$ are solutions of the same system and the general solution of $\dot{x} = Ax$ is
    \begin{equation*}
        X(t) = k_1 Re(\phi(t)) + k_2 Im(\phi(t))
    \end{equation*}
}
\ex[]{}{
    $\begin{cases*}
        \dot{x} = 5x + \frac{5}{2}y\\
        \dot{y} = -4x -y
    \end{cases*}$\\

    Step 1: Find A matrix
    \begin{equation*}
        A = \left[\begin{array}{cc}
            5 & \frac{5}{2}\\
            -4 & -1
        \end{array}\right]
    \end{equation*}
    Step 2: Find eigenvalues
    \begin{gather*}
        P(\lambda) = \left|\begin{array}{cc}
            5-\lambda & \frac{5}{2}\\
            -4 & -1-\lambda
        \end{array}\right| = (5-\lambda)(-1-\lambda)+10 = \lambda^2-4\lambda+5\\
        P(\lambda) = 0 \Leftrightarrow \lambda = \frac{4\pm\sqrt{16-4(1)(5)}}{2}\Leftrightarrow \lambda = 2\pm i
    \end{gather*}
    Step 3: Find eigenvectors for $\lambda = 2+i$
    \begin{gather*}
        \left(\begin{array}{cc}
            5-2-i & \frac{5}{2}\\
            -4 & -1-2-i
        \end{array}\right)\left(\begin{array}{c}
            x\\
            y
        \end{array}\right) = \left(\begin{array}{c}
            0\\
            0
        \end{array}\right) \Leftrightarrow\\
        \left(\begin{array}{cc}
            3-i & \frac{5}{2}\\
            -4 & -3-i
        \end{array}\right)\left(\begin{array}{c}
            x\\
            y
        \end{array}\right) = \left(\begin{array}{c}
            0\\
            0
        \end{array}\right) \Leftrightarrow\\
        \begin{cases*}
            (3-i)x + \frac{5}{2}y = 0\\
            -4x -3y-iy = 0
        \end{cases*}\Leftrightarrow\\
        y = \frac{2}{5}(i-3)x\\
        \Rightarrow E_{2+i} = \,<(5, 2i-6)>
    \end{gather*}
    General solution in $\bbC$:
    \begin{align*}
        X(t) & = e^{(2+i)t}\left(\begin{array}{c}
            5\\
            2i-6
        \end{array}\right)\\
        & = e^{2t}e^{it}\left(\begin{array}{c}
            5\\
            2i-6
        \end{array}\right)\\
        & = e^{2t}(\cos(t)+i\sin(t))\left(\begin{array}{c}
            5\\
            2i-6
        \end{array}\right)\\
        & = \left(\begin{array}{c}
            5e^{2t}(\cos(t)+i\sin(t))\\
            (2i-6)e^{2t}(\cos(t)+i\sin(t))
        \end{array}\right)\\
        & = \left(\begin{array}{c}
            5e^{2t}\cos(t) + i5e^{2t}\sin(t)\\
            e^{2t}(2i\cos(t)-2\sin(t)-6\cos(t)-6i\sin(t))
        \end{array}\right)\\
        & = \left(\begin{array}{c}
            5e^{2t}\cos(t) + i5e^{2t}\sin(t)\\
            e^{2t}(2\sin(t)-6\cos(t)) + e^{2t}(2\cos(t)-6\sin(t))i
        \end{array}\right)
    \end{align*}
    Final general solution:
    \begin{equation*}
        X(t) = k_1\left(\begin{array}{c}
            5e^{2t}\cos(t)\\
            e^{2t}(2-\sin(t)-6\cos(t))
        \end{array}\right) + k_2\left(\begin{array}{c}
            5e^{2t}\sin(t)\\
            e^{2t}(2\cos(t)-6\sin(t))
        \end{array}\right)\quad k_1,k_2\in\bbR
    \end{equation*}
}

\section{Equilibria and stability}
General setup of the question,
\begin{equation*}
    \dot{X} = f(x)
\end{equation*}
where,
\begin{itemize}
    \item $x\in\bbR$, dependent on $t$
    \item $f : \bbR^n\to\bbR^n$, a vector field in $\bbR^n$
\end{itemize}
\ex[]{}{
    \begin{enumerate}
        \item $\begin{cases*}
            \dot{x} = 3x+6y\\
            \dot{y} = 7x-y
        \end{cases*}$\\

        $f(x,y) = (3x+6y, 7x-y)\to$ vector field in $\bbR^2$
        \item $\begin{cases*}
            \dot{x} = e^xy-z\\
            \dot{y} = y^2-x^2\sin(xy)\\
            \dot{z} = 3x+y^2
        \end{cases*}$\\

        $f(x,y,z) = (e^xy-z, y^2-x^2\sin(xy), 3x+y^2)\to$ vector field in $\bbR^3$
        \item $\begin{cases*}
            \dot{x} = xe^t\\
            \dot{y} = y^2\cos(xy)
        \end{cases*}$\\

        $f(x,y) = (xe^t, y^2\cos(xy))\to$ non-autonomous vector field in $\bbR^2$ because the first term depends on $t$
    \end{enumerate}
}
\dfn[]{Equilibria}{
    \begin{equation}
        \dot{X} = f(x)\quad ,x_0\in\bbR^n
    \end{equation}
    We say that $x_0$ is an equilibria of (3.6) if $f(x_0) = 0$
}
\ex[]{Equilibria}{
    \begin{enumerate}
        \item $\dot{x} = 4x$
        \begin{gather*}
            f(x) = 4x\\
            f(x) = 0 \Leftrightarrow x=0
        \end{gather*}
        Equilibria: $x=0$
        \item $\dot{x} = x(1-x)$
        \begin{gather*}
            f(x) = x(1-x)\to\text{ logistic law}
        \end{gather*}
        Equilibria: $x=0$ and $x=1$
        \item $\begin{cases*}
            \dot{x} = 4x+7y\\
            \dot{y} = -x + 8y
        \end{cases*}$
        \begin{gather*}
            F(x,y) = (4x+7y, -x + 8y)\\
            F(x,y) = \vec{0} \Leftrightarrow \begin{cases*}
                x = 0\\
                y = 0
            \end{cases*}
        \end{gather*}
        Equilibria: $(0,0)$
        \item $\begin{cases*}
            \dot{x} = x+x^2+xy^2\\
            \dot{y} = y + y^3
        \end{cases*}$
        \begin{equation*}
            \begin{cases*}
                x+x^2+xy^2 = 0\\
                y + y^3 = 0
            \end{cases*} \Leftrightarrow \begin{cases*}
                x+x^2+xy^2 = 0\\
                y(1+y^2) = 0
            \end{cases*} \Leftrightarrow \begin{cases*}
                x+x^2+xy^2 = 0\\
                y = 0
            \end{cases*}
        \end{equation*}
        Plugging in $y=0$ into the first equation we get,
        \begin{equation*}
            x + x^2 = 0 \Leftrightarrow x(1+x) = 0 \Leftrightarrow x = 0 \vee x = -1
        \end{equation*}
        Equilibria: $(0,0)$ and $(-1,0)$
    \end{enumerate}
}

\dfn[]{Phase-portrait}{
    Phase-portraits of differential equations is a geometric picture of the solution in the dependent variable when $t$, the independent variable, varies.
}
\ex[]{Phase-portrait}{
    \begin{enumerate}
        \item $\dot{x} = 4x$
        \begin{equation*}
            x(t) = ke^{4t},\; k\in\bbR
        \end{equation*}
        \begin{center}
            \includegraphics[scale=0.4]{Images/32.png}
            \includegraphics[scale=0.3]{Images/33.png}
        \end{center}
        Equilibrium corresponds to a point.
        \item $\dot{x} = x(1-x)$
        \begin{center}
            \includegraphics[scale=0.4]{Images/34.png}
            \includegraphics[scale=0.3]{Images/35.png}
        \end{center}
        \item $\dot{x} = Ax,\;A\in M_{2\times2}(\bbR)$ where $A$ has two real positive eigenvalues, and $\lambda_1\neq\lambda_2$
        \begin{equation*}
            \left(\begin{array}{c}
                x(t)\\
                y(t)
            \end{array}\right) = c_1e^{\lambda_1t}v_1 + c_2e^{\lambda_2t}v_2,\; \lambda_1>\lambda_2
        \end{equation*}
        \begin{center}
            \includegraphics[scale=0.4]{Images/36.png}
        \end{center}
    \end{enumerate}
}
\textbf{Case 1}: $A$ has two real eigenvalues, $\lambda_1<0<\lambda_2$
\begin{center}
    \includegraphics[scale=0.4]{Images/37.png}
\end{center}

\textbf{Case 2}: $A$ has a real eigenvalues $\lambda$ with geometric multiplicity of $1$, $\lambda>0$.
\begin{center}
    \includegraphics[scale=0.5]{Images/38.png}
\end{center}

\textbf{Case 3}: $A$ has complex eigenvalues $\alpha\pm i\omega,\;\alpha<0$
\begin{center}
    \includegraphics[scale=0.5]{Images/39.png}
\end{center}

\ex[]{Phase-portrait}{
    \begin{enumerate}
        \item $\begin{cases*}
            \dot{x} = x+2y\\
            \dot{y} = x+y
        \end{cases*}$
        \begin{equation*}
            A = \left[\begin{array}{cc}
                1 & 2\\
                1 & 1
            \end{array}\right]
        \end{equation*}
        \begin{itemize}
            \item Eigenvalues: $\underbrace{1+\sqrt{2}}_{>0}\quad\underbrace{1-\sqrt{2}}_{<0}$
            \item Eigenvectors: $(\sqrt{2},1)\; (-\sqrt{2}, 1)$
        \end{itemize}
        \begin{center}
            \includegraphics[scale=0.5]{Images/40.png}
        \end{center}
        \item $\begin{cases*}
            \dot{x} = -x\\
            \dot{y} = x-y
        \end{cases*}$
        \begin{equation*}
            A = \left[\begin{array}{cc}
                -1 & 0\\
                1 & -1
            \end{array}\right]
        \end{equation*}
        \begin{itemize}
            \item Eigenvalues: $\lambda = 0$
            \item Eigenvectors: $(0,1)$
        \end{itemize}
        To find the direction of the graph,
        \begin{equation*}
            (x,y) = (1,0) \Leftrightarrow(\dot{x}, \dot{y}) = (-1,1)
        \end{equation*}
        \begin{center}
            \includegraphics[scale=0.4]{Images/41.png}
        \end{center}
        So we get the final graph,
        \begin{center}
            \includegraphics[scale=0.5]{Images/42.png}
        \end{center}
        \item $\begin{cases*}
            \dot{x} = 2x+y\\
            \dot{y} = -2x+4y
        \end{cases*}$
        \begin{itemize}
            \item Eigenvalues: $\lambda = 3\pm i$
            \item $e^{3t}(\cdots)$ positive so going away from the center spiralling
        \end{itemize}
        Again to find the direction of the graph,
        \begin{equation*}
            (x,y) = (1,0) \Leftrightarrow (\dot{x}, \dot{y}) = (2,-2)
        \end{equation*}
        \begin{center}
            \includegraphics[scale=0.4]{Images/43.png}
        \end{center}
        So we get the final graph,
        \begin{center}
            \includegraphics[scale=0.5]{Images/44.png}
        \end{center}
    \end{enumerate}
}

Suppose we have a system $\begin{cases*}
    \dot{x} = x^2+x+xy^2\\
    \dot{y} = y+y^3
\end{cases*}$, with respect to the system, we know the equilibrium. 
But it is very difficult to trace the phase portrait. 
However, it is possible to classify the behavior of the solution near the equilibria.
\begin{center}
    \includegraphics[scale=0.3]{Images/45.png}
\end{center}
\dfn[]{Stability according to Lyapunov}{
    Conditions:
    \begin{itemize}
        \item $X = F(x)$
        \item $x_0$ is an equilibria
    \end{itemize}
    Then,
    \begin{enumerate}
        \item We say that $x_0$ is stable if there exist two neighborhooods of $x_0$, $u$ and $v$ such that $u\subset v$,
        and for all $x\in u$, $\phi(t,x)\in v$, $\forall t\in\bbR_0^+$.
        \item We say that $x_0$ is asymptotically stable if $x_0$ is stable and for all $x$ in a small neighborhood of $x_0$,
        its solution tends to $x_0$.
        \item We say that $x_0$ is unstable if it is not stable.
    \end{enumerate}
}
\thm[]{Lyapunov}{
    Condition:
    \begin{itemize}
        \item $x_0$ is an equilibrium of $X=F(x)$
    \end{itemize}
    Then,
    \begin{enumerate}
        \item If all equilibrium of $\partial f|_{x_0}$ have negative real part, then $x_0$ is asymptotically stable.
        \item If at least one eigenvalue has positive real part then $x_0$ is unstable.
    \end{enumerate}
}

%\dfn{Definition Topic}{Definition Statement}
%\thm{Theorem Name}{Theorem Statement}
%\cor[cori]{Corollary Name}{Corollary Statement}
%\lem{Lemma Name}{Lemma Statement}
%\clm{Claim Name}{Claim Statement}
%\ex{Example Name}{Example explained}
%\opn{Open Question Name}{Question Statement}
%\pr{Question Name}{Question Statement}
%\nt{Special Note}
%\wc{Wrong Concept topic}{Explanation}